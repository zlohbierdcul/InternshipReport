\chapter*{Abkürzungsverzeichnis}
\begin{acronym}
    \acro{scg}[SCG]{Sales Category Group} 
    \acroplural{scg}[SCGs]{Sales Category Groups}
\end{acronym}
Eine \ac{scg} ist ein kundeninterne Nummer die Produktgruppen kategoriesiert. 

\begin{acronym}
    \acro{sapui5}[SAPUI5]{SAP User Interface 5}
\end{acronym}
\acs{sapui5} ist ein Frontend Framework von SAP.

\begin{acronym}
    \acro{cap}[SAP CAP]{SAP Cloud Application Programming Model}
\end{acronym}
\acs{cap} ist ein Backend Framework von SAP.

\begin{acronym}
    \acro{ide}[IDE]{Integrated Development Environment}
    \acroplural{ide}[IDEs]{Integrated Development Environment}
\end{acronym}
Eine \ac{ide} ist eine Softwareanwendung, welche Entwicklern beim Schreiben von Programmen unterstützen soll.

\begin{acronym}
    \acro{oop}[OOP]{Objektorientierte Programmierung}
\end{acronym}
\ac{oop} ist ein Modell der Computerprogrammierung, bei dem das Softwaredesign auf Daten oder Objekten basiert und nicht auf Funktionen und Logik.

\begin{acronym} 
    \acro{sql}[SQL]{Structured Query Language}
\end{acronym}
\acs{sql} ist eine Datenbanksprache zur Definition von Datenstrukturen in relationalen Datenbanken