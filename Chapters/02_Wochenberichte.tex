\chapter{Wochenberichte}
\label{ch:wochenberichte}

\section{Einarbeitung in das \acs{sapui5} Frontend Framework}
In meiner ersten Praktikumswoche begann mit einem kurzen technischen Onboarding, indem mir mein Equipment bereitgestellt und alle sovanta-Accounts eingerichtet wurden. Danach habe ich mich intensiv mit dem \acs{sapui5}-Framework auseinandergesetzt. Ziel war es, eine umfassende Einführung in die Grundlagen dieses Frameworks zu erhalten. Dabei lag der Fokus auf dem Verständnis von \acs{sapui5}-spezifischen Konzepten wie Data Binding, Fragments, Views und Routing. \\\\
Die Einarbeitung begann ich mit einer Analyse der Dokumentation und Online-Ressourcen, um ein solides Fundament zu legen. Dabei kamen insbesondere Fragen zum effektiven Einsatz von Data Binding und zur korrekten Implementierung von Fragments auf. \\\\
Die Herausforderung bestand darin, die theoretischen Kenntnisse in der praktischen Anwendung zu vertiefen. Durch aktives Experimentieren und Umsetzen kleiner Beispiele gelang es mir, die Zusammenhänge besser zu verstehen. Schwierigkeiten traten vor allem bei der korrekten Integration von Data Binding-Funktionalitäten auf, wobei ich auf Online-Foren und Tutorials zurückgriff, um Lösungsansätze zu finden. \\\\

\section{Vertiefung der \acs{sapui5} Kenntnisse und \acs{cap} Backend Framework}
Die zweite Woche meines Praktikums stand ganz im Zeichen der Vertiefung meines Verständnisses für das \acs{sapui5}-Frontend-Framework und dem dazugehörigen \acs{cap} Backend-Framework. \\\\
Aufgrund von Problemen mit dem Zugang zu der Codebase des Projektes, dem ich zugeteilt wurde, verzögerte sich die Einarbeitung in die Codebase, an der ich die nächsten Monate arbeiten werde. Diese Zeit nutzte ich jedoch effektiv, indem ich mich eigenständig den fortgeschrittenen Konzepten von \acs{sapui5} und der den Grundlagen von \acs{cap} widmete. Hierbei fing ich an die Dokumentation für das \acs{cap} Framework zu lesen und erste kleine Projekte aufzusetzen. \\\\
Die größte Herausforderung dieser Woche war es, die erworbenen Kenntnisse in der Praxis anzuwenden und eine Verbindung zwischen einer eigenen \acs{sapui5}-Anwendung und dem \acs{cap} Backend herzustellen und vor allem das Konzept des OData Datenmodells zu verstehen und zu implementieren, welches für SAP-Anwendungen essenziell ist. \\\\
Durch kontinuierliche Experimente und die Analyse von Beispielcode gelang es mir jedoch, ein tieferes Verständnis für dieses Datenmodell zu gewinnen und dieses Verständnis in einer ersten Anwendung mit einem \acs{sapui5} Frontend und \acs{cap} Backend umzusetzen. \\\\

\section{Einarbeitung in die Project-Codebase und erste eigene Tickets}
In der dritten Woche konnte ich endlich auf die Codebase des Projekts zugreifen, was mir die Einarbeitung in das laufende Projekt ermöglichte. Nachdem ich mich mit der Codebase etwas vertraut gemacht hatte, konnte ich mich auch direkt aktiv in das laufende Projekt einbringen, indem ich meine ersten eigenen Tickets bearbeitete. Diese Tickets beinhalteten die Implementierung verschiedener Features und Anpassungen, um die Funktionalität und Benutzerfreundlichkeit der Anwendung zu verbessern. \\\\
Die Hauptaufgabe diese Woche bestand darin, einen neuen Filter und eine neue Spalte, für bestehende Werte aus der Datenbank, zur Tabelle aller Kaufprojekte auf der Übersichtsseite hinzuzufügen. Zusätzlich zu dieser größeren Aufgabe bearbeitete ich auch kleinere Tickets, wie das Ändern von Dropdown-Optionen, die Anpassung von Abkürzungen in der gesamten Anwendung und das automatische Befüllen des "Währung"-Felds beim Erstellen eines bestimmten neuen Kaufprojektes. \\\\
Die größte Herausforderung bestand darin, die komplexe Codebase und Architektur eines schon so großen Projektes zu verstehen und dann an den richtigen Stellen Änderungen zu machen, um die Ergebnisse zu erzielen, die gefordert waren. Diese Heraufforderung konnte ich durch viel ausprobieren und befragen meines Betreuers, zumindest teilweise, bewältigen. \\\\

\section{Erstes Sprint Review und weitere Bearbeitung von Tickets}
Diese Woche lief ähnlich wie letzte Woche und ich arbeitete weiter an eigenen Tickets. Dazu kam dann allerdings auch, dass in dieser Woche das Sprint Review anstand, in dem ich dann die Ergebnisse, die ich in diesem Sprint erreicht habe, vor dem Kunden präsentieren sollte. \\\\
Für das Sprint Review habe ich mich vorher mit meinem Betreuer zusammengesetzt und er hat mir erklärt, wie das Review abläuft und wie er sich darauf vorbereitet. Anhand dieser Informationen habe ich mir dann einen Plan gemacht, was ich wie vorstellen möchte. Dafür habe ich dann auf dem Development System Beispiel Kaufprojekte angelegt, um daran die Änderungen besser vorstellen zu können und einen reibungslosen Ablauf zu garantieren. \\\\
Nach dem Review folgte dann das Sprint Planning, in dem mir dann eine neue, etwas größere, Aufgabe zugeteilt wurde. Die Aufgabe besteht darin, in der ganzen Anwendung die Länder und Ländercodes in eine, von dem Kunden vorgegebene, einheitliche Reihenfolge zu bringen. Dafür musste ich zuerst eine Liste mit allen Vorkommnissen von Ländern in der Anmeldung machen und deren Ursprung im Code herausfinden. \\\\
Diese Aufgabe zu lösen, stellte sich als größere Herausforderung heraus als eigentlich gedacht, da die Ländercodes an viele Stellen auch als zusammengesetzter String in den Datenbanktabellen der Kaufprojekte stehen und so diese Werte, bei bestehenden Projekten, nicht so einfach neu sortiert werden können. Für dieses Problem konnte ich diese Woche allerdings noch keine Lösung finden. \\\\

\section{Reihenfolge der Länder in der ganzen Anwendung ändern}
Diese Woche widmete ich mich vollkommen der Aufgabe eine Lösung für die neue Sortierung aller Ländervorkommnisse in der Anwendung zu finden. \\\\
Das größte Problem war hierbei, dass Kaufprojekte zum einen so genannte countryCodes beinhalteten, welche eine Zugehörigkeit zu diesen Ländern aussagt. Aber auch ein countryLabel, welches ein im Backend berechneter String ist, welcher die countryCodes mit einem Trennzeichen miteinander verknüpft und in der Datenbank speichert. Da es keine Option war, in der Datenbank alle countryLabel, mit der neuen Reihenfolge zu aktualisieren, musste ich hier im Backend die Rückgabe der Projekte modifizieren. Bevor diese dann an das Frontend gesendet wurden, mussten diese, getrennt, neu sortiert und dann wieder mit dem richtigen Trennzeichen zusammenfügt werden. \\\\
Ein weiteres Problem war, dass nicht nur Projekte eine Zughörigkeit zu einem oder mehreren Ländern haben, sondern es viele verschiedene Entitäten gibt, welche einen Ländercode gespeichert haben. Dieser Code war allerdings in vielen Fällen unter einem anderen Namen in den Entitäten zu finden. Das machte es schwierig, eine einzelne Funktion zum Sortieren all dieser Sonderfälle zu erstellen. Dafür musste ich die Sonderfälle separat abfragen und je nach Entität ein anderes Attribut zum Sortieren verwenden. \\\\

\section{Fertigstellung der Länderreihenfolge und eine neue Aufgabe}
Am Anfang dieser Woche habe ich mich Hauptsächlich damit beschäftigt, die letzten Fehler, die beim Testen des Features, mit dem ich mich die letzte Woche beschäftigt habe, aufgekommen sind, zu beheben. \\\\
Das größte Problem war, dass das Sortieren in manchen Fällen nicht so funktioniert hat wie es sollte. Das lag daran, dass wir uns dazu entschieden haben, sogenannte Streams, welche mehreren Ländern zugehörig waren, anhand des ersten Landes in der Liste der Länder zu sortieren. Das hatte zur Folge, dass, wenn es mehrere Streams in einem Kaufprojekt gibt, welche mit demselben Land anfingen und eine unterschiedliche Anzahl an Ländern hatten, die Sortierung nicht mehr richtig funktioniert hat. Um dieses Problem zu beheben, musste die Sortierfunktion angepasst werden. Hier entschied ich mich, zusammen mit meinem Betreuer, dazu die Länder, die mit demselben Land beginnen, anhand der Länge der Länderliste zu sortieren. \\
Am Mittwoch war dann unser Sprint Review, in dem ich wieder meine Ergebnisse der letzten zwei Wochen vor dem Project Ownern von dem Kunden vorstellen durfte. \\\\
Im Sprint Planning habe ich dann wieder eine neue Aufgabe bekommen. Diesmal sollte ich eine neue Funktionalität in dem, vor kurzem erst hinzugefügten, Admin-UI entwickeln. Da ein ähnliches Feature bereits existierte, konnte ich mich sehr stark daran orientieren und so schnell das neue Feature implementieren. Das hatte allerdings zur Folge, dass viel duplizierter Code entstanden ist, den es zu reduzieren galt. Dafür musste ich viele Funktionen abstrahieren, welche von den alten Funktionalitäten und von dem neu implementierten genutzt wurden oder nahezu identisch waren. \\\\

\section{Bugfixing und ein neues Feature}
Diese Woche fing damit an, dass ich Bugs beheben musste, welche unserem Tester aufgefallen sind. Dies waren allerdings alles nur kleine Bugs die schnell behoben werden konnten oder für die es schon einmal an einer anderen stelle aufgetreten sind und somit schon eine Lösung dafür existierte. Daher konnte ich alle Bugs am Montag bereits beheben und Dienstag mit der Implementierung eines neuen Features beginnen. \\\\
Für das neue Feature sollte ein neues Feld auf der Kaufprojekt Übersichtsseite hinzugefügt werden. Dieses Feld soll es den Benutzern erlauben, bei Kaufprojekten für Textil Produkte, mehr Informationen über die Materialien zu hinterlegen. Die Anforderungen waren, dass das Feld eine Mehrfachauswahl ermöglicht, also mehrere Werte ausgewählt werden können und dass es nur für bestimmte Kaufprojektarten und Benutzerrollen sichtbar ist. \\\\
Um dieses Feature umzusetzen, musste ich Änderungen am Front- und Backend vornehmen. Die Änderungen im Frontend waren schnell umgesetzt, da es dort nur darum ging ein neues Dropdown Feld hinzuzufügen und an ein Feld in der Datenbank zu binden. Dieses Feld musste dann nur noch unter bestimmten Konditionen ausgeblendet werden. Dank dem OData data-binding von \acs{sapui5} und \acs{cap} war das sehr schnell umzusetzen. \\\\
Die größten Änderungen passierten im Backend, da dort eine neue Entität erstellt werden musste und diese mit den möglichen Auswahlmöglichkeiten für das Feld gefüllt werden musste. Außerdem musste ein neues one-to-many Feld in einer Datenbank Tabelle angelegt werden. \\\\
Nachdem dieses Feature fertig war und von unserem Tester abgenommen wurde, habe ich angefangen Unittest zu schreiben, da es für den aktuellen Sprint keine weiteren Tickets zu bearbeiten gibt. \\\\

\section{Unittesting und Review Präsentation}
Diese Woche begann mit dem Quarterly Review, zu dem der Kunde uns eingeladen hat. In diesem 3 Stunden Meeting wurden die Fortschritte und Erfolge des letzten Quartals der verschieden Softwareprodukte von dem Kunden vorgestellt. Da ein paar dieser Produkte von sovanta entwickelt werden – darunter auch das BuyingCockpit, an dem ich beteiligt bin – war unser Team auch zu diesem Meeting eingeladen. \\\\
Am Dienstag habe ich mich auf unser Sprint Review vorbereitet, welches am Nachmittag stattfand. Hier sollte ich wieder die Features vorstellen, die ich in diesem Sprint implementiert habe. \\\\
Neben den beiden Reviews habe ich mich hauptsächlich mit den Unittests beschäftigt, mit denen ich in der letzten Woche angefangen habe. Für das Unittesting musste ich mich jedoch zuerst mit den Vorgehensweisen meines Teams vertraut machen, denn für das Mocken der Datenbank Aufrufe wurden von dem Team zum Beispiel Funktionen geschrieben, um das Mocken einfacher zu machen. \\\\
Nachdem ich mit den Unittests, an denen ich gearbeitet habe, fertig war, habe ich die Aufgabe bekommen Regressionstest durchzuführen. Da ein Teamkollege in den letzten Wochen unser Backendframework \acs{cap} auf eine neue Version zu updaten, mussten wir nun sicher gehen, dass die Anwendung mit der neuen Version noch genau so funktioniert wie vorher. Daher mussten wir alle Funktionen der Anwendung manuelle testen und dies dokumentieren. Unser Haupttester hat dafür eine Testdatei erstellt, in der die verschiedenen Testfälle, die getestet werden sollen, aufgelistet sind. Damit habe ich dann Freitag angefangen. \\\\

\section{Regression- und Unittests}
Diese Woche stand ganz im Zeichen des Testens. Am Anfang der Woche habe ich mich hauptsächlich mit den Regressionstests, mit denen ich schon in der letzten Woche begonnen habe, beschäftigt. Hierbei sind mir einige kleinere Fehler aufgefallen, über die ich dann mit unserem Tester diskutiert habe, ob für diese Fehler ein Bug-ticket erstellt werden sollte und wie kritische diese sind. Danach habe ich dann für die relevanten Fehler ein Bug-ticket erstellt, damit dieser dann so schnell wie möglich behoben werden kann. \\\\
Ein Bug, den ich gefunden habe, war ein blockierender Bug, weshalb ich die Regressionstest fürs erste pausieren musste. In ein paar Meetings mit meinen Kollegen haben wir dann diesen Fehler investigiert und versucht eine Lösung für das Problem zu finden. Schlussendlich hat dann ein Seniorentwickler das Problem identifizieren und beheben können. \\\\
Danach konnte ich dann die Regressionstests für die neue \acs{cap}-Version erfolgreich abschließen. \\\\
Da in ein paar Wochen der Release der neuen Version bevor steht, werden keine neuen Features mehr implementiert, weshalb ich mich den Rest der Woche nur noch mit weiteren Unittests beschäftigt habe. \\\\

\section{Weitere Unittests und Beginn der Testphase}
Diese Woche verlief sehr ähnlich zur letzten Woche. Am Anfang der Woche habe ich weiter an Unittests gearbeitet. Inzwischen habe ich die Vorgehensweise meines Teams deutlich besser verstanden und konnte so deutlich schneller und effizienter die Tests schreiben. Außerdem bekam ich durch das Unittesting ein noch besseres Verständnis der Architektur und der Codebase, da ich für das Testen gezwungen war die Funktionen richtig zu verstehen und auch welche Rolle diese im Großen und Ganzen spielen. \\\\
Mitte der Woche hatten wir dann unser letztes Review vor dem Release der neuen Version, da nun die Testphase beginnt.  \\\\
Als erste Instanz in der Testphase, sollte unser Team Regressionstest durchführen und alle Features manuell testen, bevor die UAT-Tests mit Benutzern des Kunden durchgeführt werden. \\\\
Hierfür wurden die verschiedenen Benutzer- und Ländergruppen zwischen allen Teammitgliedern aufgeteilt und bekamen dann Testfälle, die wir dann testen sollten. Mir wurden dann zwei Ländergruppen zugeteilt, die ich dann auf dem PRE-System testen sollte. Mit der ersten Ländergruppe beschäftigte ich mich dann den Rest der Woche. \\\\

\section{Bugfixing und Unittests}
Diese Woche hat damit angefangen, dass ich einen Bug welches mit einem Feature, welches ich vor ein paar Wochen implementiert habe, zusammenhing, beheben musste. Das Problem war, dass das Filtern auf der Übersichtsseite für manche Felder nicht mehr richtig funktionierte. Dies passierte, da mit einem neuen Feature, welches wir implementiert haben, die Funktionsweise wie wir standardmäßig die Projekte filtern, grundlegend verändert wurde. Somit funktionierten einige individuelle Filter nicht mehr und mussten an die neue funktionsweiße angepasst werden. \\\\
Nachdem ich diesen Bug beheben konnte, setzte ich mich wieder daran neue Unittests für verschiedene Dateien zu schreiben. Das habe ich dann den Rest der Woche gemacht und auch die Woche damit beendet. \\\\

\section{Komplizierteres Unittesting}
Diese ganze Woche war geprägt von weiteren Unittests. Jedoch war die Datei, welche ich diese Woche getestet habe, deutlich anspruchsvoller zu testen. Das Problem war, dass dies eine große Datei war mit vielen großen Funktionen, aber jedoch nur eine Funktion exportiert wurde und der Rest lokal war. Somit konnte ich nicht jede Funktion separat testen, sondern musste alle Funktionen indirekt über die eine exportierte Funktion abdecken. Das hatte zu folge, dass es deutlich schwieriger war, eine 100\% Abdeckung aller Pfade zu erreichen. \\\\
Ich find damit an den gesamten Code zu analysieren und zu verstehen war dort genau passiert und was die Aufgabe des Codes ist. Nachdem ich das gemacht habe, konnte ich mir einen Plan machen, was für Fälle alle eintreten können, und welche getestet werden müssen. Als das erledigt war, musste ich mir überlegen, welche Daten ich mocken muss, um die verschiedenen Testfälle abzudecken. Nun musste ich nur noch alles zusammenfügen und die Testfälle mit den Mock Daten implementieren. \\\\
Am Ende der Woche konnte ich dann den Merge Request für diese etwas aufwendigeren Unittests erstellen. \\\\

\section{Unittesting und der Start einer neuen Version}
Am Anfang dieser Woche habe ich mich hauptsächlich mit einigen Unittests beschäftigt, um die restliche Zeit in diesem Sprint sinnvoll zu nutzen. Außerdem hatten wir am Montag ein Meeting, in dem die Ziele für die neue Version der Software besprochen wurden. \\\\
Am Mittwoch hat dann der neue Sprint, und damit auch die Entwicklung für die neue Version begonnen. Mir wurde dann ein neues Ticket zugeteilt, in dem es darum geht, ein neues Feld auf der Länderebene zu erstellen. Dieses Feld soll einen Einfluss auf den Workflow des Projektes haben, dafür muss das Feld mit an den Workflow Modulator übergeben werden. Diese Woche habe ich mich hauptsächlich damit beschäftigt herauszufinden, was geändert werden muss, um das Feature zu implementieren. \\\\
Am Donnerstag hatte ich noch meinen Onboarding-Day bei sovanta, an dem wurden uns Infos zu sovanta gegeben, Workshops gehalten und wir konnten viele neue Leute kennen lernen. \\\\

\section{Probleme bei der Feature Implementierung}
In dieser Woche habe ich mich zum großen Teil damit beschäftigt, das Feature, an dem ich in der letzten Woche abgefangen habe, fertig zu stellen. \\\\
Beim Versuch das Feature zu implementieren, ist mir allerdings ein Problem aufgefallen. Das Problem war, dass das Feld laut den Akzeptanzkriterien im Frontend drei Auswahlmöglichkeiten haben sollte, „Ja“, „Nein“ und „-„ (leer). Da allerdings im Workflow Modulator von dem anderen Team das Feld mit einem Boolean modelliert wurde, machte es das schwierig drei Auswahlmöglichkeiten zu implementieren. \\\\
Um dies zu lösen hatten wir nur zwei Möglichkeiten. Entweder das Feld auf der Anwendungsseite und der Workflow Modulator Seite als eine Assoziation zu modellieren. Dafür hätte jedoch das Team, welches für den Workflow Modulator zuständig ist, größere Änderungen vornehmen müssen. Die andere Möglichkeit, für welche wir uns dann, nach Absprache mit den verantwortlichen Personen, auch entschieden haben, war es das Feld nur auf der Anwendungsseite als eine Assoziation zu modellieren und dann bevor es an den Workflow Modulator gesendet wir ein einen Boolean zu mappen. \\\\
Nachdem wir diesen Lösungsansatz gefunden haben, war die Implementierung schnell umgesetzt und konnte dann auf das DEV-System deployed werden, um dort noch einmal ausführlich getestet zu werden. \\\\

\section{Ende der Implementierungsphase und Begin der Regressiontests}
Nachdem ich Ende letzter Woche mein Feature für diese Version erfolgreich deployen konnte hieß es für mich jetzt wieder Unittests schreiben, bis der Sprint Mitte der Woche zu Ende war. \\\\
Am Ende des Sprints hatte wir wieder ein Review, in dem ich das Feature, an welchem ich gearbeitet habe, vorstellen durfte. Als Vorbereitung auf das Review habe ich mir im Vorfeld mehrere Testprojekte auf dem DEV-System erstellt und dort getestet, ob das Feature auch einwandfrei funktioniert und bereit für die Präsentation ist. \\\\
Mit Begin des neuen Sprints begannen dann auch die Regressiontests für die neue Version. Während der Durchführung der Tests sind mir ein paar Defekte aufgefallen, für welche ich dann ein Ticket erstellt habe und mich mit Kollegen darüber ausgetauscht habe. \\
Leider konnten wir die Regressiontests diese Woche noch nicht abschließen, da einige Features aktuell nicht testbar waren und wir darauf warten mussten, bis ein anderes Team den Fehler behoben hat. \\\\

\section{Unittesting und Beginn des Refactorings}
Diese Woche habe ich damit begonnen, dass ich die Unittests, an denen ich vor dem Beginn der Regressiontests gearbeitet habe, fortgesetzt habe. An diesen habe ich dann noch bis Mitte der Woche gearbeitet. \\\\
Nachdem ich mit den Unittests fertig haben wir beschlossen, dass wir nun mit dem Refactoring des Backend Quellcodes anfangen möchten. Dafür wurden bereits einige Tickets für Dateien oder Funktionsweisen, welche überarbeitet werden sollen, erstellt. \\\\
Mit wurde ein Ticket zugewiesen, in dem es darum ging die Logik und die Werte für das befüllen verschiedener Folmulare mit Standardwerten vom Frontend in das Backend zu verlegen. Dies sollte eine bessere Wartbarkeit dieser Standardwerte für die Zukunft zu schaffen. \\\\
Dafür musste ich mir zuerst einen Überblick verschaffen, wo diese Werte im Frontend gesetzt werden und welches Land mit welchen Werten befüllt wird, das ist nämlich für jedes Land unterschiedlich. Am Ende der Woche hatte ich dann einen guten Überblick, was geändert werden muss und welche Werte zu welchem Land gehören. \\\\

\section{Fertigstellung des Standardwert-Refactorings und weitere Refactorings}
Diese Woche fing damit an das Refactoring, mit welchem ich in der letzten Woche begonnen habe, fertig zu stellen. Nachdem ich in der letzen Woche alles ausfühlich analysiert habe, konnte ich nun mit der Implementierung beginnen. \\\\
Dafür fing ich an im Backend eine neue Entität zu erstellen, welche die Standardwerte für jedes Land beinhaltet. In dieser Entität werden zu jedem Land die entsprechenden IDs der Werte gespeichter. Da aber im Frontend nicht nur die IDs benötigt werden, sondern Teilweise auch die Werte die zu den IDs gehören mussten all diese Tabellen beim abfragen der Werte gejoint werden. \\\\
Nachdem die Änderungen im Backend fertig waren, konnte ich im Frontend eine Funktion schreiben, welche die Werte aus dem Backend, anhand der Länderzugehörigkeit des Users, abfrägt und dann das Formular mit diesen Werten auffüllt. Mit diesen Änderungen konnte das Refactoring dann auch deployed werden und ausführlich getestet werden. \\\\

\section{Refactoring und Unittests}
Diese Woche habe ich damit angefangen, eine Datei zu analysieren und zu schauen was in diese Datei überarbeitet werden könnte. Dafür habe ich mir dann eine Liste mit allen Problemen oder optimierungsmöglichkeiten erstellt um einen Überblick zu behalten und um damit das Ticket für das Refactoring zu erstellen. \\\\
Nachdem ich das Ticket für das Refactoring erstellt habe, konnte ich damit anfangen die Änderungen um zu setzen. \\\\
Dadurch das wärend dem Refactoring viele tote Codestücke gelöscht wurden, liefen die Unittests für diese Datei nicht mehr fehlerfrei durch. Nach einer Absprache mit meinem Betreuer, sind wir zu dem Entschluss gekommen, dass es am sinnvollsten ist, die Unittests für diese Datei komplett neu zu schreiben, da sich zu viel geändert hat und die Tests somit keinen Sinn mehr ergeben haben. \\\\
Somit fing ich an die Unittests für die Überarbeitete Datei komplett neu zu schreiben. \\\\

