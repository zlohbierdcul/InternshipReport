\newpage
\section{Anforderungsanalyse}
In diesem Kapitel wird eine kurze Analyse der aktuellen Situation durchgeführt und die Anforderungen an das Admin-UI aufgeführt. Zunächst wird der Ist-Zustand beschrieben, um die Probleme und Herausforderungen zu identifizieren.
Danach werden die Anforderungen aufgestellt, die zur Lösung des Problems erforderlich sind.

\subsection[Ist-Zustand]{Ist-Zustand}
Bei der Webanwendung, für die das Admin-UI entwickelt wird, handelt es sich um eine Plattform, die es dem Kunden ermöglicht, Kaufprojekte zu erstellen und den Benutzer Schritt für Schritt durch die notwendigen Aufgaben zu führen, die für die Umsetzung dieses Projekts erforderlich sind.
Jedoch gibt es ein Problem mit der Art und Weise, wie Änderungen an den Projektdaten durchgeführt werden.
Derzeit können Änderungen ausschließlich über direkte \gls{sql}-Abfragen in der Datenbank von einem Entwickler vorgenommen werden.

Ein Beispiel für diese Daten betrifft Kaufprojekte, bei denen regelmäßig administrative Aufgaben anfallen, wie z.B.:

\begin{itemize}
    \item Das Löschen von Kaufprojekten, die nicht mehr relevant sind,
    \item Das Ändern des Verantwortlichen für bestimmte Kaufprojekte, wenn sich Zuständigkeiten im Team ändern.
\end{itemize}

Das größte Problem an dieser Herangehensweise ist, dass die Nutzer des Systems vollständig auf Entwickler angewiesen sind, um selbst kleinere Änderungen durchführen zu können.
Dies führt zu unnötigen Verzögerungen im Arbeitsablauf, da die Entwickler nicht immer sofort verfügbar sind. Zudem entsteht für die Entwickler vermeidbare Arbeit, da sie Aufgaben übernehmen müssen, die prinzipiell von den Endnutzern selbst erledigt werden könnten.

Dieses Problem zeigt sich insbesondere in Situationen, in denen schnelle Anpassungen benötigt werden, um auf sich ändernde Anforderungen im Projektgeschäft zu reagieren, etwa wenn kurzfristig ein Kaufprojekt entfernt oder der zuständige Mitarbeiter aktualisiert werden muss.

Um dieses Problem zu lösen, wird in diesem Projekt eine eigenständige Admin-Seite entwickelt.
Diese Admin-Seite ist eine Erweiterung der bestehenden Anwendung und soll es den Endnutzern ermöglichen, administrative Aufgaben, wie das Löschen oder Bearbeiten von Projektdaten, ohne die Hilfe von Entwicklern durchzuführen.
So wird die Abhängigkeit von Entwicklern reduziert und der Arbeitsfluss des Kunden effizienter gestaltet.

\subsection[Anforderungen]{Anforderungen}
Dieses Kapitel befasst sich mit den Anforderungen an das Admin-UI, welche von dem Kunden zu Beginn des Projektes bereits verfasst wurden.
Die Funktionalität aller Seiten des Admin-UIs wurden in diesen Anforderungen festgehalten und dienten dann ebenfalls als Aktzeptanzkriterien.

\subsubsection[A1 - Reaktionszeit der Anwendung]{A1 - Reaktionszeit der Anwendung}

\begin{center}
    \captionof{table}{Anforderung A1 - Reaktionszeit der Anwendung \label{Tab:A1}}
    \begin{tabular}{ |p{0.2\linewidth}|p{0.8\linewidth}| } 
        \hline
        \rowcolor{lightgray}
        \textbf{A1} & \textbf{Reaktionszeit der Anwendung} \\
        \hline
        Nicht-Funktional &  Die Anwendung soll in einer angemessenen Zeit antworten.\\ 
        \hline
    \end{tabular}
\end{center}

\subsubsection[A2 - Sprache der Anwendung]{A2 - Sprache der Anwendung}

\begin{center}
    \captionof{table}{Anforderung A2 - Spache der Anwendung \label{Tab:A2}}
    \begin{tabular}{ |p{0.2\linewidth}|p{0.8\linewidth}| } 
        \hline
        \rowcolor{lightgray}
        \textbf{A2} & \textbf{Spache der Anwendung} \\
        \hline
        Nicht-Funktional &  Die Anwendung soll je nach Standort oder Einstellung des Nutzers in Deutsch oder Englisch angezeigt werden.\\ 
        \hline
    \end{tabular}
\end{center}

\subsubsection[A3 - Intelligente Vorschläge für Eingabefelder]{A3 - Intelligente Vorschläge für Eingabefelder}

\begin{center}
    \captionof{table}{Anforderung A4 - Intelligente Vorschläge für Eingabefelder \label{Tab:A3}}
    \begin{tabular}{ |p{0.2\linewidth}|p{0.8\linewidth}| } 
        \hline
        \rowcolor{lightgray}
        \textbf{A3} & \textbf{Intelligente Vorschläge für Eingabefelder} \\
        \hline
        Funktional & Eingabefelder sollen dem Nutzer Vorschläge anhand der bereits eingegeben Zeichen machen. 
        
        Die Vorschläge sollen sich bei der Eingabe oder Löschung von Zeichen aktualisieren. \\ 
        \hline
    \end{tabular}
\end{center}

\subsubsection[A4 - Eingabefelder der Seite zum Löschen der Kaufprojekte]{A4 - Eingabefelder der Seite zum Löschen der Kaufprojekte}

\begin{center}
    \captionof{table}{Anforderung A4 - Eingabefelder der Seite zum Löschen der Kaufprojekte \label{Tab:A4}}
    \begin{tabular}{ |p{0.2\linewidth}|p{0.8\linewidth}| } 
        \hline
        \rowcolor{lightgray}
        \textbf{A4} & \textbf{Eingabefelder der Seite zum Löschen der Kaufprojekte} \\
        \hline
        Funktional & Die Seite hat ein Eingabefeld, in dem eine oder mehrere ProjektIDs, die gelöscht werden sollen, eingetragen werden können. Das Feld ist ein Pflichtfeld und muss ausgefüllt werden. \\ 
        \hline
    \end{tabular}
\end{center}

\subsubsection[A5 - Eingabefelder der Seite zum Aktualisieren des Projektowners]{A5 - Eingabefelder der Seite zum Aktualisieren des Projektowners}

\begin{center}
    \captionof{table}{Anforderung A4 - Eingabefelder der Seite zum Aktualisieren des Projektowners \label{Tab:A5}}
    \begin{tabular}{ |p{0.2\linewidth}|p{0.8\linewidth}| } 
        \hline
        \rowcolor{lightgray}
        \textbf{A5} & \textbf{Eingabefelder der Seite zum Aktualisieren des Projektowners} \\
        \hline
        Funktional & Die Seite hat drei Eingabefelder:
        \begin{itemize}
            \item Der alte Projektowner (Pflichtfeld)
            \item Der neue Projektowner (Pflichtfeld)
            \item Eine oder mehrere \glspl{scg} (Optional)
        \end{itemize} 
        Pflichtfelder müssen ausgefüllt werden. 
        
        Optionale Felder müssen keinen Wert beinhalten.\\ 
        \hline
    \end{tabular}
\end{center}

\subsubsection[A6 - Eingabefelder der Seite zum Aktualisieren des Projektmanagers und der Käufergruppe]{A6 - Eingabefelder der Seite zum Aktualisieren des Projektmanagers und der Käufergruppe}

\begin{center}
    \captionof{table}{Anforderung A6 - Eingabefelder der Seite zum Aktualisieren des Projektmanagers und der Käufergruppe \label{Tab:A6}}
    \begin{tabular}{ |p{0.2\linewidth}|p{0.8\linewidth}| } 
        \hline
        \rowcolor{lightgray}
        \textbf{A6} & \textbf{Eingabefelder der Seite zum Aktualisieren des Projektmanagers und der Käufergruppe} \\
        \hline
        Funktional & Die Seite hat fünf Eingabefelder:
        \begin{itemize}
            \item Der alte Projektmanager (Pflichtfeld)
            \item Der neue Projektmanager (Pflichtfeld)
            \item Die alte Käufergruppe (Pflichtfeld)
            \item Der neue Käufergruppe (Pflichtfeld)
            \item Eine oder mehrere \glspl{scg} (Optional)
        \end{itemize} 
        Pflichtfelder müssen ausgefüllt werden.
       
        Optionale Felder müssen keinen Wert beinhalten.\\ 
        \hline
    \end{tabular}
\end{center}

\newpage

\subsubsection[A7 - Speichern und Verwerfen Aktionen]{A7 - Speichern und Verwerfen Aktionen}

\begin{center}
    \captionof{table}{Anforderung A7 - Speichern und Verwerfen Aktionsleiste \label{Tab:A7}}
    \begin{tabular}{ |p{0.2\linewidth}|p{0.8\linewidth}| } 
        \hline
        \rowcolor{lightgray}
        \textbf{A7} & \textbf{Speichern und Verwerfen Aktionsleiste} \\
        \hline
        Funktional &  Auf jeder Seite des Admin-UIs soll eine Aktionsleiste mit zwei Knöpfen vorhanden sein:
        \begin{itemize}
            \item \textbf{Speichern:} Öffnet ein Pop-Up in dem der Nutzer seine Wahl bestätigen muss. Bei Bestätigung wird die jeweilige Aktion der Seite durchgeführt. Der Knopf ist nur Verwendbar, wenn alle Pflichtfelder ausgefüllt sind.
            \item \textbf{Verwerfen:} Setzt alle Eingabefelder zurück.
        \end{itemize}\\
        \hline
    \end{tabular}
\end{center}

\begin{center}
    \captionof{table}{Anforderung A7.1 - Speicheraktion auf der Seite zum Löschen der Kaufprojekte \label{Tab:A7.1}}
    \begin{tabular}{ |p{0.2\linewidth}|p{0.8\linewidth}| } 
        \hline
        \rowcolor{lightgray}
        \textbf{A7.1} & \textbf{Speicheraktion auf der Seite zum Löschen der Kaufprojekte} \\
        \hline
        Funktional &  Wird die Speicheraktion ausgeführt sollen alle ausgewählten Kaufprojekte in der Datenbank als gelöscht markiert werden und dürfen danach nicht mehr angezeigt werden. \\
        \hline
    \end{tabular}
\end{center}

\begin{center}
    \captionof{table}{Anforderung A7.2 - Speicheraktion auf der Seite zum Aktualisieren des Projektowners \label{Tab:A7.2}}
    \begin{tabular}{ |p{0.2\linewidth}|p{0.8\linewidth}| } 
        \hline
        \rowcolor{lightgray}
        \textbf{A7.2} & \textbf{Speicheraktion auf der Seite zum Aktualisieren des Projektowners} \\
        \hline
        Funktional &  Wird die Speicheraktion ausgeführt, soll der alte Projektowner mit dem neuen Projektowner in allen Projekten ersetzt werden. Diese Aktualisierung soll auf Projekt und Länder -Ebene geschehen.

        Für den Fall, dass eine oder mehrere \glspl{scg} mit angegeben wurden, beschränkt sich die Aktualisierung auf Projekte, die mit den angegebenen \glspl{scg} übereinstimmen.\\
        \hline
    \end{tabular}
\end{center}

\newpage

\begin{center}
    \captionof{table}{Anforderung A7.3 - Speicheraktion auf der Seite zum Aktualisieren des Projektmanagers und der Käufergruppe \label{Tab:A7.3}}
    \begin{tabular}{ |p{0.2\linewidth}|p{0.8\linewidth}| } 
        \hline
        \rowcolor{lightgray}
        \textbf{A7.3} & \textbf{Speicheraktion auf der Seite zum Aktualisieren des Projektmanagers und der Käufergruppe} \\
        \hline
        Funktional &  Wird die Speicheraktion ausgeführt, soll der alte Projektmanager und die alte Käufergruppe mit dem neuen Projektmanager und der neuen Käufergruppe in allen Projekten ersetzt werden. Diese Aktualisierung soll auf Projekt und Länder -Ebene geschehen.

        Für den Fall, dass eine oder mehrere \glspl{scg} mit angegeben wurden, beschränkt sich die Aktualisierung auf Projekte, die mit den angegebenen \glspl{scg} übereinstimmen.\\
        \hline
    \end{tabular}
\end{center}