\section{Zusammenfassung}
Dieser Bericht dokumentiert die Entwicklung und Implementierung einer eigenständigen Admin-Seite für eine Webanwendung, die es Endnutzern ermöglicht, administrative Aufgaben unabhängig von technischen Entwicklern durchzuführen.
Vor der Einführung dieser Lösung waren Änderungen an den Datenbankinhalten manuell und zeitintensiv, was zu Verzögerungen und erhöhten Kosten führte.

Die neue Admin-Seite erlaubt den Nutzern, Projekte zu löschen, Informationen zu aktualisieren und andere administrative Tätigkeiten selbstständig zu erledigen.
Dabei wurde ein besonderer Fokus auf die Umstellung von JavaScript auf TypeScript im Frontend gelegt, um die Codequalität zu verbessern und den Entwicklungsprozess effizienter zu gestalten.

Die Implementierung der Admin-Seite führte zu einer spürbaren Effizienzsteigerung und einer Reduzierung der Abhängigkeit von den Entwicklern.
Diese Arbeit beleuchtet die Herausforderungen während der Entwicklung sowie die ergriffenen Lösungen und zeigt, wie die neue Admin-Seite zu einer signifikanten Verbesserung der internen Abläufe beigetragen hat.