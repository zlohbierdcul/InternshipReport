\section{Einleitung}
In Umfeld einer Webanwendung ist die plege der dazugehörigen Daten ein essenzieller Bestandteil.
Diese Pflege der Daten musste bisher manuell von den Entwicklern übernommen werden, indem dafür Änderungen direkt in der Datenbank vorgenommen wurden.
Diese Herangehensweise sorgt dafür, dass kleine Änderungen viel Zeit in anspruch nehmen und zudem auch vermeidbare Kosten verursacht.
Um dieses Problem zu lösen, wurde in diesem Projekt eine eigenständige Admin Seite entwickelt, die es den Endnutzern ermöglicht, administrative Aufgaben eigenständig und effizient durchzuführen.

Das Ziel dieser Arbeit ist es die Entwicklung und Implementierung dieser Admin Seite zu dokumentieren und die aufgetretenen Herausforderungen und deren Lösungen zu beschreiben.
Ein besonderer Fokus liegt dabei auf der Umstellung von JavaScript auf TypeScript im Frontend, um dort die Codequalität und Wartbarkeit der Codebase zu verbessern.
Die neue Admin Seite soll es den Nutzern ermöglichen, Projekte zu löschen, Informationen zu aktualisieren und andere administrative Aufgaben eigenständig durchzuführen, ohne auf die Hilfe von Entwicklern angewiesen zu sein.
