\section{Einleitung}

In Umfeld einer Webanwendung ist die Pflege der dazugehörigen Daten ein essenzieller Bestandteil.
Diese Pflege der Daten musste bisher manuell von den Entwicklern übernommen werden, indem Änderungen direkt in der Datenbank vorgenommen wurden.
Für den Kunden führte diese Abhängigkeit von den Entwicklern zu einer Reihe von Problemen.

Zum einen entstanden Verzögerungen, da selbst kleine Änderungen in den Daten, wie z.B. das Aktualisieren von Projektdetails oder das Löschen von nicht mehr benötigten Einträgen, oft Wartezeiten zur Folge hatten, bis die Entwickler verfügbar waren.
Dies war besonders problematisch bei einer hohen Auslastung der Entwickler oder bei dringenden Korrekturen.
Zum anderen verursachte dieses Vorgehen unnötige Kosten, da für jede Änderung technisches Personal hinzugezogen werden musste, auch wenn die Aufgaben relativ trivial waren.

Darüber hinaus war die Fehleranfälligkeit durch manuelle Datenbankänderungen erhöht, da Änderungen ohne Benutzeroberfläche eine präzise Kenntnis der Datenbankstruktur voraussetzten.
Jede kleine Ungenauigkeit konnte zu Dateninkonsistenzen oder anderen Problemen führen, die zusätzlichen Aufwand erforderte.

Um diese Probleme zu beseitigen und den Arbeitsfluss des Kunden zu verbessern, wurde in diesem Projekt eine eigenständige Admin-Seite entwickelt, die es den Endnutzern ermöglicht, administrative Aufgaben selbstständig und effizient durchzuführen.
Damit wird das Unternehmen des Kunden unabhängiger von Entwicklern, was zu einer schnelleren Bearbeitung von Aufgaben und geringeren Kosten führt.

Das Ziel dieser Arbeit ist es, die Entwicklung und Implementierung dieser Admin-Seite zu dokumentieren und die aufgetretenen Herausforderungen sowie deren Lösungen zu beschreiben.
Ein besonderer Fokus liegt dabei auf der Umstellung von JavaScript auf TypeScript im Frontend, um dort die Codequalität und Wartbarkeit der Codebase zu verbessern.

Die neue Admin-Seite soll es den Nutzern ermöglichen, Projekte zu löschen, Informationen zu aktualisieren und andere administrative Aufgaben eigenständig durchzuführen, ohne auf die Hilfe von Entwicklern angewiesen zu sein.
Dies trägt zu einer spürbaren Effizienzsteigerung im Betrieb des Kunden bei.