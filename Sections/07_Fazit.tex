\section{Fazit}
In diesem Projekt wurde erfolgreich eine neue Admin-Seite entwickelt, die es den Nutzern ermöglicht, administrative Aufgaben eigenständig und effizient durchzuführen.
Diese Lösung hat die Abhängigkeit von Entwicklern erheblich reduziert und die Effizienz der Arbeitsabläufe gesteigert.

Die neue Admin-Seite erlaubt es den Endnutzern, Projekte zu löschen und Informationen selbst zu aktualisieren, ohne auf technische Unterstützung angewiesen zu sein.
Dadurch können kleine Änderungen nun direkt und zeitnah umgesetzt werden, was zu einer spürbaren Effizienzsteigerung führt.

Die Entscheidung, TypeScript anstelle von JavaScript zu verwenden, hat sich als äußerst vorteilhaft erwiesen.
Die Integration von TypeScript hat nicht nur dazu beigetragen, die Codequalität zu verbessern, sondern auch den Entwicklungsprozess zu optimieren.
Durch die verbesserte Lesbarkeit und Struktur des Codes konnte schneller gearbeitet und Fehler frühzeitig erkannt werden, was den gesamten Entwicklungszyklus beschleunigt hat.

Zusammenfassend lässt sich sagen, dass die neue Admin-Seite einen bedeutenden Fortschritt für das Unternehmen des Kunden darstellt.
Sie bietet eine benutzerfreundliche Lösung für administrative Aufgaben und legt gleichzeitig den Grundstein für zukünftige Entwicklungen.
Die positiven Effekte auf die Effizienz und Kostensenkung werden langfristig zu einer verbesserten Zusammenarbeit zwischen den Nutzern und den Entwicklern führen.