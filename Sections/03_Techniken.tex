\section{Technologische Grundlagen}
\subsection[Einführung in TypeScript]{Einführung in TypeScript}
TypeScript ist eine Open-Source-Programmiersprache, welche erstmals 2012 von Microsoft veröffentlicht wurde.
TypeScript ist ein Superset von JavaScript und erweitert JavaScript um eine statische Typisierung und andere erweiterte Funktionen.
Das Ziel von TypeScript ist es, die Entwicklung von großer und komplexer JavaScript-Anwendungen einfacher und sicherer zu machen.
\subsubsection[Warum TypeScript?]{Warum TypeScript?}
JavaScript ist dynamisch typisiert, was bedeutet, dass Variablen während der Laufzeit jeden beliebigen Typ annehmen können. Das kann zu Laufzeitfehlern führen, welche schwer zu debuggen sind.
In TypeScript wird dieses Problem gelöst, indem es den Entwicklern ermöglicht, Typen explizit zu definieren, was mehrere Vorteile bietet:

\begin{itemize}
    \item \textbf{Fehlererkennung zur Entwicklungszeit:} \\   Durch die statische Typisierung werden viele Fehler bereits während dem Schreiben des Codes erkannt und nicht erst zur Laufzeit des Programms.
    \item \textbf{Verbesserte IDE-Unterstützung:} \\    Die Entwicklungsumgebung wird von TypeScript durch eine Autovervollständigung, Refactoring-Tools und IntelliSense erheblich verbessert.
    \item \textbf{Bessere Dokumentation:} \\     Durch die Typenannotation ist TypeScript Code selbstdokumentierend, was die Verständlichkeit und Wartbarkeit erhöht.
    \item \textbf{Skalierbarkeit:} \\   Die Entwicklung und Wartung großer Codebasen wird durch klare Typdefinitionen und Modulunterstützungen verbessert.  
\end{itemize}

\newpage

\subsubsection[Grundlegende Konzepte]{Grundlegende Konzepte}
\begin{itemize}
    \item \textbf{Typen:} \\
     TypeScript erweitert JavaScript um ein starkes Typensystem. Einiger der grundlegenden Typen sind:
     \begin{lstlisting}
        var isDone: boolean = true;
        var count: number = 42;
        var name: string = "Alice";
        var list: number[] = [1, 2, 3];
     \end{lstlisting}
    \item \textbf{Interfaces:} \\
    Interfaces definieren die Struktur eines Objekts und können sicherstellen, dass Objekte diese Form einhalten.
    \begin{lstlisting}
        interface Person {
            firstName: string;
            lastName: string;
        }

        function greet(person: Person) {
            return "Hello, " + person.firstName + " " + person.lastName;
        }

        let user = { firstName: "Jane", lastName: "Doe" };
        console.log(greet(user));   // "Hello, Jane Doe"
    \end{lstlisting}
    \item \textbf{Klassen:} \\
    TypeScript unterstützt objektorientierte Programmierung durch Klassen, was Vererbung, Kapselung und andere OOP-Konzepte ermöglicht.
    \begin{lstlisting}
        class Animal {
            name: string;

            constructor(name: string) {
                this.name = name;
            }
            move(distance: number = 0) {
                console.log(`${this.name} moved ${distance}m.`);
            }
        }

        class Dog extends Animal {
            bark() {
                console.log("Woof! Woof!");
            }
        }

        let dog = new Dog("Buddy");
        dog.bark();
        dog.move(10);
    \end{lstlisting}
    \item \textbf{Generics:} \\
    Generics ermöglichen die Erstellung wiederverwendbarer Komponenten, die mit verschiedenen Typen arbeiten können.
    \begin{lstlisting}
        function identity<T>(arg: T): T {
            return arg;
        }

        let output1 = identity<string>("myString");
        let output2 = identity<number>(100);
    \end{lstlisting}
\end{itemize}

\subsection[Unterschiede zwischen JavaScript und TypeScript]{Unterschiede zwischen JavaScript und TypeScript}

\subsection[Vorstellung weiterer verwendeter Technologien und Frameworks]{Vorstellung weiterer verwendeter Technologien und Frameworks}