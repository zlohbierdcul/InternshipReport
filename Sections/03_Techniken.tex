\section{Technologische Grundlagen}
\subsection[Einführung in TypeScript]{Einführung in TypeScript}
TypeScript ist eine Open-Source-Programmiersprache, welche erstmals 2012 von Microsoft veröffentlicht wurde.
TypeScript ist ein Superset von JavaScript und erweitert JavaScript um eine statische Typisierung und andere erweiterte Funktionen.
Das Ziel von TypeScript ist es, die Entwicklung von großer und komplexer JavaScript-Anwendungen einfacher und sicherer zu machen.
\subsubsection[Warum TypeScript?]{Warum TypeScript?}
JavaScript ist dynamisch typisiert, was bedeutet, dass Variablen während der Laufzeit jeden beliebigen Typ annehmen können. Das kann zu Laufzeitfehlern führen, welche schwer zu debuggen sind.
In TypeScript wird dieses Problem gelöst, indem es den Entwicklern ermöglicht, Typen explizit zu definieren, was mehrere Vorteile bietet:

\begin{itemize}
    \item \textbf{Fehlererkennung zur Entwicklungszeit:} \\   Durch die statische Typisierung werden viele Fehler bereits während dem Schreiben des Codes erkannt und nicht erst zur Laufzeit des Programms.
    \item \textbf{Verbesserte IDE-Unterstützung:} \\    Die Entwicklungsumgebung wird von TypeScript durch eine Autovervollständigung, Refactoring-Tools und IntelliSense erheblich verbessert.
    \item \textbf{Bessere Dokumentation:} \\     Durch die Typenannotation ist TypeScript Code selbstdokumentierend, was die Verständlichkeit und Wartbarkeit erhöht.
    \item \textbf{Skalierbarkeit:} \\   Die Entwicklung und Wartung großer Codebasen wird durch klare Typdefinitionen und Modulunterstützungen verbessert.  
\end{itemize}

\newpage

\subsubsection[Grundlegende Konzepte]{Grundlegende Konzepte}
\begin{itemize}
    \item \textbf{Typen:} \\
     TypeScript erweitert JavaScript um ein starkes Typensystem. Einiger der grundlegenden Typen sind:
     \begin{lstlisting}[caption={Beispiel: Typen in TypeScript}]
        var isDone: boolean = true;
        var count: number = 42;
        var name: string = "Alice";
        var list: number[] = [1, 2, 3];
     \end{lstlisting}
    \item \textbf{Interfaces:} \\
    Interfaces definieren die Struktur eines Objekts und können sicherstellen, dass Objekte diese Form einhalten.
    \begin{lstlisting}[caption={Beispiel: Interfaces in TypeScript}]
        interface Person {
            firstName: string;
            lastName: string;
        }

        function greet(person: Person) {
            return "Hello, " + person.firstName + " " + person.lastName;
        }

        let user = { firstName: "Jane", lastName: "Doe" };
        console.log(greet(user));   // "Hello, Jane Doe"
    \end{lstlisting}
    \item \textbf{Klassen:} \\
    TypeScript unterstützt objektorientierte Programmierung durch Klassen, was Vererbung, Kapselung und andere OOP-Konzepte ermöglicht.
    \begin{lstlisting}[caption={Beispiel: Klassen in TypeScript}]
        class Animal {
            name: string;

            constructor(name: string) {
                this.name = name;
            }
            move(distance: number = 0) {
                console.log(`${this.name} moved ${distance}m.`);
            }
        }

        class Dog extends Animal {
            bark() {
                console.log("Woof! Woof!");
            }
        }

        let dog = new Dog("Buddy");
        dog.bark();
        dog.move(10);
    \end{lstlisting}
    \item \textbf{Generics:} \\
    Generics ermöglichen die Erstellung wiederverwendbarer Komponenten, die mit verschiedenen Typen arbeiten können.
    \begin{lstlisting}[caption={Beispiel: Generics in TypeScript}]
        function identity<T>(arg: T): T {
            return arg;
        }

        let output1 = identity<string>("myString");
        let output2 = identity<number>(100);
    \end{lstlisting}
\end{itemize}

\subsection[Unterschiede zwischen JavaScript und TypeScript]{Unterschiede zwischen JavaScript und TypeScript}
Da TypeScript auf JavaScript aufbaut, gibt es viele Gemeinsamkeiten zwischen den beiden Sprachen.
Jedoch gibt es zwei wesentliche Punkte, in denen sich die Sprachen grundsätzlich unterscheiden:

\begin{enumerate}
    \item \textbf{Statische Typisierung} \\
    In JavaScript werden Variablen dynamisch typisiert und Typen zur Laufzeit überprüft. In TypeScript werden diese statisch typisiert, was es Entwicklern erlaubt, Typen explizit zu deklarieren und zur Komplierzeit zu überprüfen.
    Dies reduziert die Wahrscheinlichkeit von Typfehlern und verbessert die Codequalität.
    \begin{lstlisting}
        // JavaScript
        let count = 42; // Dynamisch typisiert

        // TypeScript
        let count: number = 42; // Statisch typisiert
    \end{lstlisting}
    \item \textbf{Typanmerkungen und -infernz} \\
    In TypeScript können Typen explizit angegeben werden oder von der TypeScript-Engine automatisch inferiert werden.
    Diese Inferenz ermöglicht eine flexible Typisierung, welche sowohl die Sicherheit als auch die Bequemlichkeit erhöht.
    \begin{lstlisting}
        let message: string = "Hello, World!"; // Explizite Typanmerkung
        let count = 42; // Typinferenz: TypeScript erkennt automatisch, dass count vom Typ number ist
    \end{lstlisting}
\end{enumerate}

\subsection[Vorstellung weiterer verwendeter Technologien und Frameworks]{Vorstellung weiterer verwendeter Technologien und Frameworks}

\subsubsection[Das SAP UI5 Framework]{Das SAP UI5 Framework}
SAP UI5 ist ein auf JavaScript basierendes UI-Framework, um plattformübergreifende Webanwendungen für Unternehmen effizient zu entwickeln.

Es wurde 2011 von SAP entwicket und wird bis heute weiterentwickelt. Seit 2013 gibt es von dem Framework auch eine open-source variante namens OpenUI5. Der Hauptunterschied der beiden Frameworks ist allerdings nur die Lizenz.

Das Hauptziel des Frameworks ist es eine konsistente, reaktionsschnelle und intuitive Benutzererfahrung über alle SAP-Anwendungen hinweg zu bieten, die sowohl auf Desktops als auch auf mobilen Geräten reibungslos funktioniert.

\subsubsection[Grundlegende Architektur von SAP UI5]{Grundlegende Architektur von SAP UI5}
SAP UI5 basiert auf der Model-View-Controller (MVC) Architektur, ein gängiges Entwurfsmuster, welches eine klare Trennung der verschiedenen Anwendungskomponenten gewährleistet:
\begin{itemize}
    \item \textbf{Model:} \\
    Das Model repräsentiert die Datenstruktur und -logik der Anwendung. Es beinhaltet die Daten, welche aus unterschiedlichen Quellen geladen werden können, wie zum Beispiel OData-Services, JSON-Dateien oder anderen APIs.
    \item \textbf{View:} \\
    Die View beschreibt das Layout und Design der Benutzeroberfläche. SAP UI5 bietet verschiedene Ansätze für die Definition der Views, darunter XML, HTML, JSON und JavaScript.
    Der beliebteste Ansatz sind die XML-Views, da sie die Strukturierung und Wiederverwendbarkeit fördern.
    \item \textbf{Constroller:} \\
    Der Coltroller verarbeitet die Benutzereingaben und steuert die Geschäftslogik.
    Er dient als Vermittler zwischen dem Model und der View, indem er die Daten abruft und an die View weiterleitet oder Benutzereingaben verarbeitet und entsprechende Aktionen 
\end{itemize}

Diese Trennung ermöglicht eine bessere Wartbarkeit und Skalierbarkeit von Anwendungen und erleichtert die Zusammenarbeit in Entwicklerteams.

\subsubsection[Vorteile und Schlüsselmerkmale von SAP UI5]{Vorteile und Schlüsselmerkmale von SAP UI5}
SAP UI5 bietet eine Reihe von Vroteilen und Funktionen, die es zu einem leistungsstarken Framework für die Entwicklung von Unternehmensanwendungen machen:

\begin{itemize}
    \item \textbf{Responsive Design:} \\
    Durch SAP UI5 werden Anwendungen automatisch an verschiedene Bildschirmgrößen angepasst, wodurch sie auf Desktop-Computern und mobilen Geräten gut funktionieren.
    \item \textbf{UI-Komponenten:} \\
    Das Framework bietet eine umfangreiche Bibliothek von vorgefertigten UI-Komponenten wie Tabellen, Diagrammen, Formularen und vielem mehr.
    Diese Komponenten sind dafür ausgelegt, eine einheitliche Benutzererfahrung zu gewährleisten und Anwendungen effizienter entwickeln zu können.
    \item \textbf{Internationalisierung und Lokalisierung:} \\
    Mit SAP UI5 können Anwendung mithilfe des sogenannten i18n-Models an verschiedene Sprachen und Regionen angepasst werden. 
    \item \textbf{Integration mit SAP-Systemen:} \\
    Das Framework ist speziell darauf ausgelegt, nahtlos mit SAP-Backend-Systemen zu arbeiten, insbesondere mit OData-Services.
\end{itemize}

\subsubsection[SAP UI5 in Verbindung mit TypeScript]{SAP UI5 in Verbindung mit TypeScript}
SAP UI5 in TypeScript ist eine relativ neue Erweiterung zu UI5, die 2021 von SAP veröffentlicht wurde und immer weiter verbessert wird.
Es bringt alle Vorteile die TypeScript gegenüber JavaScript mitbringt zu UI5, was die Entwicklung der Anwendungen stark verbessert.

\subsubsection[Unterschied zwischen JavaScript und TypeScript in UI5 anhand eines Beispiels]{Unterschied zwischen JavaScript und TypeScript in UI5 anhand eines Beispiels}
Im Folgenden werden die Unterschiede eines UI5 BaseControllers in JavaScript und TypeScript gezeigt. 

\textbf{JavaScript:} \\
In JavaScript werden UI5 Controller über eine interne Funktion definiert. 
Diese erwartet ein Array mit Pfaden zu Controllern, die importiert werden sollen, und einer Callback-Funktion, welche die Funktktionalität des Controllers defniert.
\begin{lstlisting}[caption={Beispiel: JavaScript BaseController.js}]
    sap.ui.define([
        "sap/ui/core/mvc/Controller"
    ], function (Controller) {
        "use strict";

        return Controller.extend("de.example.controller.BaseController", {
            getRouter: function () {
                return this.getOwnerComponent().getRouter();
            },
            getModel: function (sName) {
                return this.getView().getModel(sName);
            }
        })
    })
\end{lstlisting}

\textbf{TypeScript:} \\
Anders als in JavaScript werden Controller in TypeScript über Klassen definiert. 
Damit der TypeScript Kompiller die Klasse jedoch in einen UI5 Controller konvertieren kann, muss ein Kommentar mit dem passenden namespace hinzugefügt werden.
Controller können mit einem TypeScript Import importiert werden.
\begin{lstlisting}[caption={Beispiel: TypeScript BaseController.ts}, ]
    import Controller from "sap/ui/core/mvc/Controller";

    /**
     * @namespace de.example.controller
     */
    export default class BaseController extends Controller {
        public getRouter(): Router {
            return (this.getOwnerComponent() as UIComponent).getRouter();
        }

        public getModel(name?: string): Model {
            this.getView().getModel(name);
        }
    }
\end{lstlisting}
